\documentclass[a4paper]{article}
\usepackage[12pt]{extsizes}
\usepackage{fullpage}
\usepackage{cmap}
\usepackage{graphicx}
\usepackage{multirow}
\usepackage{amsmath,amsthm,amsfonts,amssymb,amscd}
\usepackage{mathrsfs}
\usepackage{listings}
\usepackage[utf8]{inputenc}
\usepackage[T1]{fontenc}
\usepackage[english, russian]{babel}
\usepackage{minted}


\usepackage{fancyhdr} % для колонтитулов
\pagestyle{fancy}
\fancyhf{}
\setlength{\headheight}{0pt}
\renewcommand{\headrulewidth}{0pt}

\newcommand{\anonsection}[1]{ \section*{#1} \addcontentsline{toc}{section}{\numberline {}#1}}

\begin{document}
	\begin{center}
		\textbf{
			МИНИСТЕРСТВО ОБРАЗОВАНИЯ РЕСПУБЛИКИ БЕЛАРУСЬ \\
			БЕЛОРУССКИЙ ГОСУДАРСТВЕННЫЙ УНИВЕРСИТЕТ \\
			Факультет прикладной математики и информатики
		} \\
		
		\vspace{1em}
		
		\text {
			Кафедра математического моделирования и анализа данных
		}
	\end{center}
	
		\vspace{10em}
		
	\begin{center}
		Лабораторная работа 1 \\
		\vspace{1em}
		\textbf{
			Численные методы решения краевых задач \\
			для ОДУ второго порядка \\
		}
	 	\vspace{1em}
		Вариант 11
		
	\end{center}
	
	\vspace{10em}
	
	\begin{minipage}{0.4\textwidth}
		\begin{center}
			\textbf{Выполнил} \\
			\textit{Цуранов Никита Васильевич} \\
			3 курс, 7 группа
		\end{center}
	\end{minipage}
	\hfill
	\begin{minipage}{0.4\textwidth}
		\begin{center}
			\textbf{Преподаватель} \\
			\textit{Радкевич Елена Владимировна} \\
		\end{center}
	\end{minipage}
	
	\fancyfoot[C]{\large{Минск, 2020}}

	\newpage
	
	\fancyfoot[C]{\thepage}
	
	
	\tableofcontents
	\newpage
	
	
	\anonsection{Постановка задачи}
	
	$$ 
	\begin{cases}
		(k(x) u'(x))' - q(x) u(x) = -f(x),	0 \le x \le 1 \\
		k(0) u'(0) = a_0 u(0) - b_0, \\
		-k(1) u'(1) = a_1 u(1) - b_1, \\	
	\end{cases}
	$$
	
	$$ 
	\begin{matrix}
		f(x) & k(x) & q(x) & a_0 & b_0 & a_1 & b_1  \\
		x \sin(2x) & \cos^2(x) & \sin(2x) & 0 & -1 & 1 & \cos^2(1)
	\end{matrix}
	$$
	
	\begin{enumerate}
		\item Аппроксимировать поставленную задачу разностной схемой 2-го порядка на минимальном шаблоне. Для повышения порядка аппроксимации граничных условий выделить главный член погрешности и заменить его, используя следующий вид исходного дифференциального уравнения: 
		$$k(x)u''(x) + k'(x)u'(x) - q(x)u(x) = -f(x)$$
		\item При помощи интегро-интерполяционного метода построить консервативную разностную схему, вычисляя интегралы по формуле средних прямоугольников
		\item Аппроксимировать исходную задачу вариационно-разностным методом, для вычисления коэффициентов используя квадратурную формулу трапеций
		\item Методом разностной прогонки реализовать полученные разностные схемы с шагом 0.1, провести сравнительный анализ решений при разных значениях шага
	\end{enumerate}
	
	\section{Метод повышения порядка точности}
	
	Используя шаблон $\{x - h, x, x + h\}$ для аппроксимации схемой второго порядка построим разностный оператор:
	\begin{large}
		$$ 	k(x) y_{lr} + k'(x) y_m - q(x) y = -f(x) $$
		$$ \text{где } y_l, y_m, y_r \text{ --- левая, центральная, правая производные}$$
	\end{large}
	Условия на границах можно аппроксимировать только первым порядком, т.к. $x - h$ не входит в сетку в точке 0, а $x + h$ --- в 1:
	$$ 
	\begin{matrix}
	y_r(0) = 1 &&& y_l(1) = 1 - \frac{u(1)}{\cos^2(1)}
	\end{matrix}
	$$
	Для повышения порядка разложим погрешность аппроксимации в ряд Тейлора:
	$$
	\begin{matrix}
	u(h) = u(0) + u'(0)h + \frac{u''(0)h^2}{2} + o(h^2) &&&
	u(1- h) = u(1) - u'(1)h + \frac{u''(1)h^2}{2} + o(h^2)
	\end{matrix}
	$$
	$$
		u'(0) - \frac{u(h) - u(0)}{h} 
		= u'(0) - \frac{u(0) + u'(0)h + \frac{u''(0)h^2}{2} + o(h^2) - u(0)}{h} 
		= -\frac{u''(0)h}{2} + O(h^2)
	$$
	$$  
		u'(1) - \frac{u(1) - u(1 - h)}{h} 
		= u'(1) - \frac{u(1) - u(1) + u'(1)h - \frac{u''(1)h^2}{2} + o(h^2)}{h} 
		= \frac{u''(1)h}{2} + O(h^2)
	$$
	$$ k(x) u''(x) + k'(x) u'(x) - q(x) u(x) = -f(x) \Rightarrow u''(x) = \frac{k'(x) u'(x) + q(x) u(x) - f(x)}{k(x)} $$
	$$ 
		u'(0) \approx u_r(0) - \frac{u''(0)h}{2} =
		u_r(0) - \frac{k'(0) (a_0 u(0) - b_0) + q(0) u(0) k(0) - f(0) k(0)}{k(0)^2}
	$$
	$$ 
		u'(1) \approx u_l(1) + \frac{u''(1)h}{2} =
		u_l(1) - \frac{k'(1) (a_1 u(1) - b_1) - q(1) u(1) k(1) + f(1) k(1)}{k(1)^2}
	$$
	
	Получим следующую схему:
	$$ 
	\begin{cases}
	k(x) y_{lr} + k'(x) y_m - q(x) y = -f(x), x \in w_h \\
	y_r(0) k(0)^2 - k'(0) (a_0 y(0) - b_0) - q(0) y(0) k(0) + f(0) k(0) = (a_0 y(0) - b_0 ) k(0), \\
	y_r(1) k(1)^2 - k'(1) (a_1 y(1) - b_1) + q(1) y(1) k(1) - f(1) k(1) = -(a_1 y(1) - b_1 ) k(1).
	\end{cases}
	$$
	
	Или в индексном виде:
	\begin{large}
	$$ 
	\begin{cases}
	k_i \frac{y_{i+1} -2y_i +  y_{i-1}}{h^2} + k'_i  \frac{y_{i + 1} - y_{i - 1}}{2h} - q_i y_i = -f_i, i = \overline{1, N-1} \\
	\frac{y_1 - y_0}{h} k_0^2 - k'_0 (a_0 y_0 - b_0) - q_0 y_0 k_0 + f_0 k_0 = (a_0 y_0 - b_0 ) k_0, \\
	\frac{y_N - y_{N-1}}{h} k_N^2 - k'_N (a_1 y_N - b_1) + q_N y_N k_N - f_N k_N = -(a_1 y_N - b_1) k_N.
	\end{cases}
	$$
		
	$$ 
	\begin{cases}
	\frac{2k_i - k'_i h}{2h^2} y_{i-1} -
	\frac{2k_i + q_i h^2}{h^2} y_i +
	\frac{2k_i + k'_i h}{2h^2} y_{i+1}  = -f_i, i = \overline{1, N-1} \\
	(-\frac{k_0^2}{h} - k'_0 a_0 - q_0 k_0 - a_0 k_0) y_0 +
	\frac{k_0^2}{h} y_1 = 
	-f_0 k_0 + b_0 (k'_0 - k_0),\\
	-\frac{k_N^2}{h} y_{N - 1} + 
	(\frac{k_N^2}{h} - k'_N a_1 + q_N k_N + a_1 k_N) y_N = 
	f_N k_N + b_1 (k'_N + k_N).\\
	\end{cases}
	$$
	\end{large}	
	
	Полученную систему можно свести к системе линейных уравнений. Матрица системы трехдиагональная, а значит применим метод прогонки.
	
	\newpage
	
	\section{Метод Баланса}	
	Проинтегрировав исходное уравнение на отрезке $[x_{i-0.5}, x_{i+0.5}] $ получим:
	$$ 
	k(x) u'(x)\Big|_{x_{i-0.5}}^{x_{i+0.5}} - \int_{x_{i-0.5}}^{x_{i+0.5}} q(x) u(x) dx = -\int_{x_{i-0.5}}^{x_{i+0.5}} f(x) dx, 
	i = \overline{1, N-1}
	$$
	Используя квадратурную формулу средних прямоугольников найдем:
	$$ 
	\int_{x_{i-0.5}}^{x_{i+0.5}} f(x) dx \approx 
	f(x_i) h
	$$
	
	$$
	\int_{x_{i-0.5}}^{x_{i+0.5}} q(x) u(x) dx \approx
	q(x_i) u(x_i) h
	$$
	
	Рассмотрим первое слагаемое:
	$$ 
	k(x) u'(x)\Big|_{x_{i-0.5}}^{x_{i+0.5}} =
	k(x_{i+0.5}) u'(x_{i+0.5}) - k(x_{i-0.5}) u'(x_{i-0.5})
	$$
	$$ 	u'(x_{i-0.5}) \approx u_m(x_{i-0.5}) = \frac{u(x_{i}) - u(x_{i-1})}{h} \text{ (рассматривая с половинным шагом)} $$
	
	Тогда все полученное подставим в исходное уравнение и пусть $ c_i =  k(x_{i - 0.5}) $:
	\begin{large}
		$$ 
		c_{i + 1} \frac{y_{i+1} - y_i}{h} -
		c_{i} \frac{y_i - y_{i-1}}{h} -
		q_i y_i h = - f_i h, 
		i = \overline{1, N-1}
		$$
		$$ 
		c_i y_{i-1} -
		(q_i h^2 + c_i + c_{i+1}) y_i +
		c_{i + 1} y_{i+1} = - f_i h^2, 
		i = \overline{1, N-1}
		$$
	\end{large}
	
	Рассмотрим аппроксимацию краевых условий:
	$$
	\begin{matrix}
	\vspace{1em}
	k(0) u'(0) = a_0 u(0) - b_0, &&
	\begin{matrix}
	k(x) u'(x)|_{0}^{\frac{h}{2}} - \int_{0}^{\frac{h}{2}}q(x) u(x) dx = -\int_{0}^{\frac{h}{2}} f(x) dx \\
	c_1 \frac{y_1 - y_0}{h} - a_0 y_0 + b_0 -
	\frac{h}{2} q(\frac{h}{4}) y_0 =
	-\frac{h}{2} f(\frac{h}{4}) \\
	\Rightarrow -(\frac{c_1}{h} + a_0 + q(\frac{h}{4}) \frac{h}{2}) y_0 +
	\frac{c_1}{h} y_1 = -b_0 - \frac{h}{2} f(\frac{h}{4})
	\end{matrix} \\
	-k(1) u'(1) = a_1 u(1) - b_1, &&
	\begin{matrix}
	k(x) u'(x)|_{1-\frac{h}{2}}^{1} - \int_{1-\frac{h}{2}}^{1} q(x) u(x) dx = -\int_{1-\frac{h}{2}}^{1} f(x) dx \\
	-a_1 y_N + b_1 -c_N \frac{y_N - y_{N-1}}{h} -
	\frac{h}{2} q(1 - \frac{h}{4}) y_N = 
	-\frac{h}{2} f(1 - \frac{h}{4}) \\
	\Rightarrow \frac{c_N}{h} y_{N-1} -
	(\frac{c_N}{h} + a_1 + q(1 - \frac{h}{4})\frac{h}{2} ) y_N =
	- b_1 -\frac{h}{2} f(1 - \frac{h}{4})
	\end{matrix} \\
	\end{matrix}
	$$
	
	Исходя из всех полученных уравнений мы опять можем построить систему линейных уравнений с трехдиагональной матрицей системы и решить методом прогонки.
	
	\newpage
	
	\section{Метод Ритца}
	
	Метод Ритца позволяет искать минимум функционала, тогда исходная задача будет эквивалентна отысканию минимума функционала:
	$$
		J(u) = \frac{1}{2} \Big(a_0 u^2(0) + a_1 u^2(1) +
		\int_{0}^{1} (k(x) (u'(x))^2 + q(x) u^2(x))dx\Big) -
		b_0 u(0) - b_1 u(1) - \int_{0}^{1} f(x)u(x)dx
	$$
	Суть поиска заключается в построении подпространства и поиск минимума в нем. Чтобы построить разностную схему нужно подобрать координатные функции специальным образом.
	
	\begin{center}
		\includegraphics[scale=0.5]{img.png}
	\end{center}

	Тогда система примет вид:
	\begin{large} 
	$$
	\begin{cases}
		\frac{c_i}{h^2} y_{i-1} - (\frac{c_i + c_{i + 1}}{h^2} + d_i) y_i + \frac{c_{i+1}}{h^2} y_{i+1} = -\phi_i, i = \overline{1, N - 1} \\
		-(a_0 + \frac{c_1}{h} + \frac{h}{2} d_0) y_0 +
		\frac{c_1}{h} y_1 = -b_0 - \frac{h}{2} \phi_0 \\
		\frac{c_N}{h} y_{N - 1} -
		(a_1 + \frac{c_N}{h} + \frac{h}{2} d_N) y_N  =
		-b_1 - \frac{h}{2} \phi_N
	\end{cases}\text{, где:}
	$$ 
	$$ 
	\begin{matrix}
		\vspace{0.5em}
		c_i = \frac{1}{h} \big( \int_{x_{i-1}}^{x_i} k(x) dx - \int_{x_{i-1}}^{x_i} q(x) (x_i - x)(x - x_{i-1}) dx \big), i = \overline{1, N} \\ 
		d_i = \frac{1}{h^2} \big( \int_{x_{i-1}}^{x_i} q(x)(x - x_i) dx - \int_{x_i}^{x_{i + 1}} q(x) (x_{i+1} - x) dx \big), i = \overline{1, N-1} \\
		\vspace{0.5em}
		d_0 = \frac{2}{h^2} \int_{0}^{h} q(x)(h - x) dx;
		d_N = \frac{2}{h^2} \int_{1-h}^{1} q(x)(x - 1 + h) dx \\
		\phi_i = \frac{1}{h^2} \big( \int_{x_{i-1}}^{x_i} f(x)(x - x_i) dx - \int_{x_i}^{x_{i + 1}} f(x) (x_{i+1} - x) dx \big), i = \overline{1, N-1} \\
		\phi_0 = \frac{2}{h^2} \int_{0}^{h} f(x)(h - x) dx;
		\phi_N = \frac{2}{h^2} \int_{1-h}^{1} f(x)(x - 1 + h) dx \\
	\end{matrix}
	$$
	\end{large}

	Интегралы найдем численно по методу трапеций:
	$$
		\int_{a}^{b} f(x)dx \approx
	 	\frac{b - a}{2} \big(f(a) + f(b) \big)
	$$
	
	\begin{large} 
		$$ 
		\begin{matrix}
		\vspace{0.5em}
		c_i = \frac{k(x_{i-1}) + k(x_i)}{2} = \frac{k_{i-1} + k_i}{2}, i = \overline{1, N} \\ 
		d_i = q(x) = q_i, i = \overline{1, N-1} \\
		\vspace{0.5em}
		d_0 = - q_0; d_N =  q_N \\ 
		\phi_i = f(x) = f_i, i = \overline{1, N-1} \\
		\vspace{0.5em}
		\phi_0 = - f_0; \phi_N =  f_N \\
		\end{matrix}
		$$
	\end{large}	
	
	\newpage
	
	\anonsection{Листинг программы:}
	
	\begin{minted}[linenos=true, tabsize=4]{Python}
import numpy as np
from numpy import power, sin, cos, tan
from scipy.linalg import norm, solve_banded

true_ans = np.linspace(-1, 0, 11)
f = lambda x: x * sin(2 * x)
k = lambda x: power(cos(x), 2)
q = lambda x: sin(2 * x)
dk = lambda x: -sin(2 * x)
a_0 = 0
b_0 = -1
a_1 = 1
b_1 = cos(1)**2

x = np.linspace(0, 1, 11)[1: -1]
i_range = np.arange(1, x.size - 1)
h = 0.1
f_v = f(x)
k_v = k(x)
q_v = q(x)
dk_v = dk(x)

def Solve(A_0_0, A_0_1, A_n_nm1, A_n_n,
          A_below, A_in, A_above,
          b_beg, b_mid, b_end):
	A = np.array([
		[0, A_0_1, *A_above],
		[A_0_0, *A_in, A_n_n],
		[*A_below, A_n_nm1, 0],
	])
	
	b = [b_beg, *b_mid, b_end]
	
	solution = solve_banded((1, 1), A, b)
	print('Solution:')
	for i in solution:
	print('{:.5f}'.format(i), end=' ')
	disc = solution - true_ans
	print('\nDiscrepancy:')
	for i in disc:
	print('{:+.1e}'.format(i), end=' ')
	print('\nRelative error: {:.2e}'.format(norm(disc) / norm(solution)))
	print()

def IncreasingOrderMethod():
	print('Increasing order method: ')
	Solve(-k(0) ** 2 / h - dk(0) * a_0 - q(0) * k(0) - a_0 * k(0), k(0) ** 2 / h,
	      -k(1) ** 2 / h, k(1) ** 2 / h - dk(1) * a_1 + q(1) * k(1) + a_1 * k(1),
	      (2 * k_v - dk_v * h) / h ** 2 / 2,
	      -(2 * k_v + q_v * h ** 2) / h ** 2,
	      (2 * k_v + dk_v * h) / h ** 2 / 2,
	      -f(0) * k(0) + b_0 * (dk(0) - k(0)),
	      -f_v,
	      f(1) * k(1) + b_1 * (dk(1) + k(1)))

def BalanceMethod():
	print('Balance method: ')
	c = np.append(k(x - h / 2), k(1 - h / 2))
	Solve(-(c[0] / h + a_0 + q(h / 4) * h / 2), c[0] / h,
	      c[-1] / h, -(c[-1] / h + a_1 + q(1 - h / 4) * h / 2),
	      c[:-1], -(q_v * h ** 2 + c[1:] + c[:-1]), c[1:],
	      -b_0 - h / 2 * f(h / 4),
	      -f_v * h**2,
	      -b_1 - h / 2 * f(1 - h / 4))

def RitzMethod():
	print('Ritz method: ')
	x_ext = np.linspace(0, 1, 11)
	c = (k(x_ext[:-1]) + k(x_ext[1:])) / 2
	Solve(-(a_0 + c[0] / h - h * q(0) / 2), c[0] / h,
	      c[-1] / h, -(a_1 + c[-1] / h + h * q(1) / 2),
          c[:-1] / h**2, -(c[:-1] + c[1:]) / h**2 - q_v, c[1:] / h**2,
	      -b_0 + h * f(0) / 2, -f_v, -b_1 - h * f(1) / 2)

print('True solution:')
for i in true_ans:
print('{:+.5f}'.format(i), end=' ')
print('\n')

IncreasingOrderMethod()
BalanceMethod()
RitzMethod()
	
	\end{minted}
	
	Использовался метод прогонки из библиотеки numpy. На вход принимает матрицу из диагоналей и их количество (в нашем случае одна над и одна под)
	
	\newpage
	
	\anonsection{Вывод программы}
	Вывод для h = 0.1\\
	\includegraphics[scale=0.6]{img2.png}
	\\ Вывод для h = 0.01 \\
	\includegraphics[scale=0.6]{img3.png}
	
	\anonsection{Выводы}
	
	Каждый из методов обладает одинаковыми вычислительной сложностью и порядком точности, а именно вторым, что можно увидеть по результатам. Но на практике для этой задачи оказался лучшим метод повышения порядка аппроксимации, т.к. его невязка связанна с машинной погрешностью, потому что ошибка в 15 знаке.
	
	В случае моего варианта посчитать производную от k(x) не составило труда, поэтому мы с могли избежать дополнительной погрешности. Хотя для более сложных функций можно было использовать численную производную. Б\'oльшую погрешность метода Ритца можно попробовать объяснить большим количеством интегралов, которые считались приближенно.
	
	При уменьшении шага, как и ожидалась возросла точность, но невязка 1 метода увеличилась из-за машинной погрешности. 
	
	\anonsection{Лист оценивания}
	\begin{tabular}{ | l | l | c |}
		\hline
		№ & Параметры оценивания & Оценка \\ 
		\hline
		1 & Оформление отчета: &  \\
		  & необходимые формулы набраны и получены самостоятельно & 3 \\
		\hline
		2 & Программирование: &  \\
		& алгоритм закодирован и отлажен самостоятельно & 3 \\
		\hline
		3 & Анализ результатов: &  \\
		& полученные результаты оценены самостоятельно & 3 \\
		\hline
	\end{tabular}\\
	\begin{tabular}{ r | l |}
	\hspace{19.3em} Баллы преподавателя & \hspace{3.35em}  \\
	\end{tabular}
	
\end{document}